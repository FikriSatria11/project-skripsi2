\chapter{KESIMPULAN DAN SARAN}

\section{Kesimpulan}
Berdasarkan hasil penelitian yang diuraikan pada bab IV, maka peneliti menarik beberapa kesimpulan sebagai berikut :
\begin{enumerate}[topsep=0pt,itemsep=0pt,partopsep=0pt, parsep=0pt]
    \item Sistem identifikasi kendaran pada pemarkiran dengan pengenalan citra plat dan pembacaan rfid berhasil diimplementasikan. Rancangan alat yang dibuat masih bersifat \textit{prototype}. Adapun alat yang digunakan yaitu Raspberry Pi, RFID, ultrasonik, servo, dan kamera.
    \item Membuat aplikasi web sebagai \textit{user interface}. Web dibuat dengan bahasa pemrograman python dan flask sebagai \textit{framework}. Web yang dibuat memiliki beberapa fitur seperti melihat slot yang terisi dan tersedia, mendaftarkan pengguna baru, menambah saldo, dan melihat informasi mengenai slot parkir.
\end{enumerate}

\section{Saran}
Adapun saran yang diberikan kepada peneliti berikutnya apabila ingin mengembangkan penelitian ini agar menjadi lebih baik adalah sebagai berikut :
\begin{enumerate}[topsep=0pt,itemsep=0pt,partopsep=0pt, parsep=0pt]
    \item Melakukan penelitian dan analisis lebih lanjut pada deteksi nomor pelat kendaraan karena pada penelitian ini, peneliti masih menggunakan API dari pihak ketiga.
    \item Menambahkan fitur tambah slot parkir agar sistem parkir menjadi lebih fleksibel.
    \item Menambahkan sensor disetiap slot parkir untuk mengetahui apakah pengendara sudah parkir ditempat yang sudah ditentukan.
\end{enumerate}