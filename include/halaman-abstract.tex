\chapter*{ABSTRACT}

\textit{The concept of the Internet of Things has been widely used in everyday life in various fields, such as agriculture, health, industry, security, and transportation. One of the problems that are often encountered in today's era is the parking system that still uses conventional methods to record the license plate number of the vehicle that will be parked. The purpose of this research is to create a system that utilizes Internet of Things technology in the transportation sector. The system in question is an automatic parking system that uses RFID reading and vehicle plate image recognition. The microcomputer used is a Raspberry Pi and the sensors used are RFID MFRC522, ultrasonic sensor HC-SR04, servo SG90, and Raspberry Pi V2 camera. This research also uses the web as a User Interface which is made with the Python and Flask programming languages as the framework.}

\textit{The results obtained are the design of tools and web that are integrated with each other. All sensors used will be connected to the RaspberryPi. The data obtained from the sensor will be stored on the database and will be displayed on the web as a user interface. The data displayed is the ID data taken from the RFID sensor and the vehicle plate number data taken from the camera. The ultrasonic sensor will take vehicle distance data and the servo functions as a parking barrier. The web created has several features such as viewing filled and available slots, registering new users, adding balances, and viewing information about parking slots.}

\begin{table}[h]
    \begin{tabular}{ p{0.17\textwidth} p{0.8\textwidth} }
        \\
        \textbf{Keywords :} & \textit{Internet of Things}, \textit{Smart Parking System}, \textit{Raspberry Pi}, \textit{Web}, \textit{Database}
    \end{tabular}
\end{table}