\chapter{PENDAHULUAN}

\section{Latar Belakang}
Revolusi Industri merupakan periode di mana terjadinya perubahan secara besar-besaran di bidang pertanian, manufaktur, tekstil dan logam, pertambangan, transportasi, teknologi, dan sosial ekonomi \thecite{azli2017a}. Pada abad ke-18, mesin uap pertama ditemukan di Inggris. Mesin uap tersebut digunakan sebagai alat tenun mekanis pertama yang dapat meningkatkan produktivitas industri tekstil. Saat itu mesin uap mulai menggantikan peralatan kerja yang awalnya bergantung pada tenaga manusia dan hewan sekaligus memulai era revolusi industri pertama yang dikenal dengan Revolusi Industri 1.0. Revolusi industri kedua ditandai dengan penemuan tenaga listrik pada awal abad ke-20. Revolusi industri ketiga ditandai oleh mesin yang dapat bergerak dan berpikir secara otomatis, yaitu komputer dan robot \thecite{rahayu2019a}. 

Di abad ke-21 revolusi industry telah masuk ke era baru. Yakni telah berada pada revolusi industri keempat atau lebih dikenal dengan Revolusi Industri 4.0. Era ini telah mengubah banyak bidang kehidupan manusia, termasuk ekonomi, dunia kerja, bahkan gaya hidup. Revolusi industri 4.0 menawarkan teknologi cerdas yang dapat terhubung dengan berbagai bidang kehidupan manusia. Revolusi industri 4.0 menerapkan \textit{Internet of Things} \textit{(IoT)} dan teknologi pada kegiatan analisis, manufaktur, robotik, komputasi canggih, \textit{artificial intellegence}, teknologi, kognitif, \textit{advance materials} dan \textit{augmented reality} dalam melaksanakan siklus operasi bisnis \thecite{suharman2019a}. Saat ini negara-negara di dunia mulai berkopetisi dalam pemanfaatan teknologi pada setiap sektor industrinya. Lalu bagaimana dengan indonesia ? Mempelajari konsep industri 4.0 untuk penerapannya di Indonesia menjadi suatu keharusan, sebab jika tidak maka industry dan manufaktur di Indonesia tidak akan dapat bersaing dengan industry dan manufaktur di negara-negara lain di dunia.

Revolusi industri 4.0 mencakup beragam teknologi canggih, seperti kecerdasan buatan (AI), \textit{wearables}, robotika canggih, \textit{3D printing}, dan \textit{Internet of Things} \textit{(IoT)}. \textit{Internet of Things} \textit{(IoT)} adalah sekenario dari suatu objek yang dapat melakukan pengiriman data/informasi melalui jaringan tanpa campur tangan manusia \thecite{limantara2017a}. Konsep \textit{Internet of Things} sudah banyak digunakan dalam kehidupan sehari-hari di berbagai bidang, seperti bidang pertanian, bidang kesehatan, bidang industri, bidang keamanan, serta bidang transportasi.

Salah satu permasalahan yang banyak dijumpai diera sekarang adalah sistem parkir yang masih menggunakan metode konvensional untuk mencatat nomor pelat kendaraan yang akan parkir dan pembayaran yang masih menggunakan uang cash. Hal ini dapat memicu kemacetan, polusi udara dan suara, dan menambah tingkat stress pengendara. Untuk mengatasi masalah tersebut, dibuatlah suatu sistem dengan memanfaatkana konsep \textit{Internet of Things} di bidang transportasi dan keamanan. Sistem yang dimaksud adalah sistem parkir otomatis yang diletakkan di tempat khusus seperti di apartemen atau di perkantoran.

Salah satu aspek dalam sistem parkir otomatis adalah identifikasi citra pelat kendaraan untuk mendapatkan data nomor pelat tanpa campur tangan manusia. Identifikasi pelat kendaraan pada sistem parkir otomatis dapat dilakukan dengan kartu RFID dan pengolahan citra pelat kendaraan. Kemampuan RFID sebagai media pengenal secara nirkabel membuat RFID sering digunakan sebagai otorisasi untuk akses ruangan dan tempat, akan tetapi penggunaan RFID masih rentan terhadap keamanan akses. Penelitian ini menggabungkan identifikasi kartu RFID dan pembacaan citra pelat kendaraan sehingga dapat meningkatkan keamanan parkir dan dapat mempersingkat waktu pencatatan nomor pelat kendaraan yang masih bersifat konvensional.

\section{Rumusan Masalah}
Berdasarkan uraian pada latar belakang masalah diatas, dapat dikemukakan pertanyaan penelitian sebagai berikut:
\begin{enumerate}[topsep=0pt,itemsep=0pt,partopsep=0pt, parsep=0pt]
    \item Bagaimana cara merancang dan membangun sistem identifikasi kendaraan dengan pengenalan citra pelat dan pembacaan RFID ?
    % \item Bagaimana cara membuat aplikasi web untuk memantau situasi lahan parkir ?
    \item Bagaimana cara membuat aplikasi web sebagai \textit{user interface} sistem parkir ?
\end{enumerate}

\section{Tujuan Penelitian}
Berdasarkan rumusan masalah, maka tujuan penelitian ini adalah :
\begin{enumerate}[topsep=0pt,itemsep=0pt,partopsep=0pt, parsep=0pt]
    \item Merancang sistem identifikasi kendaraan dengan pengenalan citra pelat dan pembacaan RFID. 
    \item Membuat aplikasi web sebagai \textit{user interface}.
    % \item Menghubungkan sistem parkir dengan web yang dibuat sebagai \textit{user interface}.
\end{enumerate}

\section{Batasan Masalah}
Batasan masalah pada penelitian ini adalah :
\begin{enumerate}[topsep=0pt,itemsep=0pt,partopsep=0pt, parsep=0pt]
    \item Alat yang dibuat bersifat prototype.
    \item Sistem parkir yang dibuat ditujukan untuk diterapkan di lingkungan berpenghuni tetap seperti apartemen atau perkantoran.
    \item Karakter pada pelat nomor harus sesuai dengan yang digunakan Samsat.
    \item Tidak melakukan analisis lebih lanjut pada deteksi nomor pelat kendaraan.
\end{enumerate}

\section{Manfaat Penelitian}
Hasil penelitian ini diharapkan dapat bermanfaat :
\begin{enumerate}[topsep=0pt,itemsep=0pt,partopsep=0pt, parsep=0pt]
    \item Menghemat waktu dan bahan bakar.
    \item Terciptanya alat yang dapat menopang kemajuan industri.
    \item Mengurangi antrian Panjang yang disebabkan oleh pencatatan nomor pelat dan pembayaran parkir yang masih konvensional.
    \item Mempermudah pengelolah parkir untuk memantau sisa kapasitas lahan parkir yang tersedia.
    \item Membantu upaya pemerintah dalam pembangunan \textit{smart city} di Indonesia.
\end{enumerate}

