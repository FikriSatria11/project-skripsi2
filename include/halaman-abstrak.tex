\chapter*{ABSTRAK}

Konsep \textit{Internet of Things} sudah banyak digunakan dalam kehidupan sehari-hari di berbagai bidang, seperti bidang pertanian, bidang kesehatan, bidang industri, bidang keamanan, serta bidang transportasi. Salah satu permasalahan yang banyak dijumpai diera sekarang adalah sistem parkir yang masih menggunakan metode konvensional untuk mencatat nomor pelat kendaraan yang akan parkir. Tujuan penelitian ini adalah untuk membuat suatu sistem yang memanfaatkan teknologi \textit{Internet of Things} di bidang transportasi. Sistem yang dimaksud adalah sistem parkir otomatis yang menggunakan pembacaan RFID dan pengenalan citra plat kendaraan. \textit{Microcomputer} yang digunakan adalah Raspberry Pi dan sensor yang digunakan adalah RFID MFRC522, sensor ultrasonik HC-SR04, servo SG90, dan kamera Raspberry Pi V2. Pada penelitian ini juga menggunakan web sebagai \textit{User Interface} yang dibuat dengan bahasa pemrograman \textit{Python} dan Flask sebagai \textit{framework}nya.

Hasil penelitian yang diperoleh adalah rancangan alat dan web yang saling terintegrasi. Semua sensor yang digunakan akan dihubungkan dengan Raspberry Pi. Data yang didapat dari sensor akan disimpan ke \textit{database} dan akan ditampilkan pada web sebagai \textit{user interface}. Data yang ditampilkan adalah data \textit{id} yang diambil dari sensor RFID dan data nomor plat kendaraan yang diambil dari kamera. Sensor ultrasonik akan mengambil data jarak kendaraan dan servo berfungsi sebagai palang parkir. Web yang dibuat memiliki beberapa fitur seperti melihat slot yang terisi dan tersedia, mendaftarkan pengguna baru, menambah saldo, dan melihat informasi mengenai slot parkir.

\begin{table}[h]
    \begin{tabular}{ p{0.17\textwidth} p{0.8\textwidth} }
        \\
        \textbf{Kata Kunci :} & \textit{Internet of Things}, Sistem \textit{Smart Parking}, Raspberry Pi, Web, \textit{Database}
    \end{tabular}
\end{table}