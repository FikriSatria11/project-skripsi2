\chapter*{KATA PENGANTAR}

Segala puji dan syukur penulis panjatkan kehadirat Allah SWT, yang senantiasa melimpahkan rahmat dan karunia-Nya. Shalawat dan salam senantiasa tercurah kepada Nabi Besar Rasulullah Muhammad SAW yang telah membawa kita ke luar dari zaman kegelapan menuju zaman yang terang benderang saat ini. 

Alhamdulillah, skripsi dengan judul "\textit{SISTEM IDENTIFIKASI KENDARAAN PADA PEMARKIRAN DENGAN PENGENALAN CITRA PLAT DAN PEMBACAAN RFID}" yang disusun sebagai salah satu syarat untuk mencapai gelar Sarjana pada program studi Sistem Informasi fakultas Matematika dan Ilmu Pengetahuan Alam Universitas Hasanuddin ini dapat diselesaikan. Walaupun dalam penulisan skripsi ini sempat terkendala dengan adanya wabah Covid-19, tetapi penulis mampu menyelesaikan pada waktu yang tepat berkat bantuan dan dukungan dari berbagai pihak.

Penulis menghanturkan ungkapan hormat dan terima kasih yang tulus kepada keluarga besar penulis terkhusus bagi Ayahanda \textbf{Satu Dani} dan Ibunda \textbf{Rismawati Amir}, yang dengan penuh kesabaran dalam mengasuh dan mendidik penulis dan tak kenal lelah dalam memanjatkan doa serta memberikan nasihat dan motivasi kepada penulis. Ucapan terima kasih juga kepada saudari tercinta \textbf{Alfiyah Gemintang Ramadhani} dan \textbf{Alya Arsyi Amdani} yang senantiasa memberikan dukungan dan doa bagi penulis dalam menyelesaikan skripsi ini.

Penulis menyadari bahwa penelitian ini dapat terselesaikan dengan adanya bantuan, bimbingan, dukungan dan motivasi dari berbagai pihak. Oleh karena itu, penulis mengungkapkan ucapan terima kasih dengan tulus kepada :

\begin{enumerate}[topsep=0pt,itemsep=0pt,partopsep=0pt, parsep=0pt]
    \item Rektor Universitas Hasanuddin, Ibu \textbf{Prof. Dr. Dwia Aries Tina Pulubuhu} beserta jajarannya.
    \item Dekan Fakultas Matematika dan Ilmu Pengetahuan Alam, \textbf{Dr. Eng. Amiruddin} beserta jajarannya.
    \item Ketua Departemen Matematika FMIPA, \textbf{Dr. Nurdin, S.Si., M.Si.}, dan juga \textbf{Dr. Muhammad Hasbi, M.Sc.} sebagai ketua Program Studi Sistem Informasi Universitas Hasanuddin.
    \item Bapak \textbf{Dr. Eng. Armin Lawi, S.Si., M.Eng.} sebagai pembimbing utama yang telah banyak memberikan arahan, ide, motivavsi serta dukungan kepada penulis dalam penyelesaian skripsi.
    \item Ibu \textbf{Musfira Putri Lukman, S.T., M.T.} yang sebagai pembimbing pertama yang senantiasa memberikan masukan dan arahan kepada penulis.
    \item Bapak \textbf{Dr. Hendra, S.Si., M.Kom.} sebagai anggota tim penguji atas saran dan kritik yang membangun pada penelitian yang telah dilakukan oleh penulis.
    \item Ibu \textbf{Nur Hilal, S.Si., M.Si.} sebagai anggota tim penguji atas saran dan kritik yang membangun pada penelitian yang telah dilakukan oleh penulis.
    \item Bapak \textbf{Supri Bin Hj. Amir, S.Si., M.Eng.} sebagai dosen pembimbing akademik yang senantiasa memberikan motivasi, dorongan, dan masukan dalam hal akademik.
    \item Seluruh Bapak dan Ibu dosen FMIPA Universitas Hasanuddin yang telah mendidik dan memberikan ilmunya sehingga penulis mampu menyelesaikan program sarjana. Serta para staf yang telah membantu dalam pengurusan berkas administrasi.
    % \item Saudara-saudara \textbf{Sunu Squad} \textbf{(Sulaeman, Muhammad Akbar Atori, Nur Ikhwan Putra Pratama, Baharuddin Kasim, Andi Rezki Muh Nur, Bagas Prasetyo, Andi Yaumil Falakh, Abdul Aziz Mubarak)} yang telah menemani penulis selama perkuliahan, saling memberi motivasi dan bantuan, meluangkan waktu dan berbagi suka-duka serta kebersamaan selama menuntut ilmu.
    \item Keluarga besar \textbf{Ilmu Komputer Unhas 2016} yang setia menemani dan penulis selama menjalani pendidikan.
    \item Kakak-kakak dan Adik-adik \textbf{Ilmu Komputer 2014, 2015, 2017, 2018} yang telah banyak membantu, semoga tetap semangat dalam mengejar impian.
    \item Rekan-rekan \textbf{KKN E-Commerce Luwu Utara} yang telah menjadi keluarga baru selama KKN dan menjadikan KKN sebagai momen yang membahagiakan.
    \item Semua pihak yang telah banyak berpartisipasi, baik secara langsung maupun tidak langsung dalam penyusunan skripsi ini yang tidak sempat penulis sebutkan satu per satu.
\end{enumerate}

Penulis menyadari bahwa skripsi ini masih jauh dari sempurna dikarenakan terbatasnya pengalaman dan pengetahuan yang dimiliki penulis. Oleh karena itu, penulis mengharapkan segala bentuk saran serta masukan bahkan kritik yang membangun dari berbagai pihak. Semoga tulisan ini memberikan manfaat kepada semua pihak yang membutuhkan dan terutama untuk penulis.

\vspace{1cm}
\begin{flushright}
    Makassar, 9 September 2021\\
    \vspace{2.5cm}
    {MUH FIKRI SATRIA AMDANI}\\
    NIM. {H13116501}
\end{flushright}